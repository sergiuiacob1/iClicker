\chapter*{Concluzii} 
\addcontentsline{toc}{chapter}{Concluzii}
Am început această lucrare de licență cu un anumit set de obiective în minte.
În acest sens am reușit să le transpun în dezvoltarea unei aplicații care poate realiza, în primul rând, sarcina de \emph{gaze tracking}.
Am reușit apoi să simulez cu succes deplasarea cursorului mouse-ului și apăsarea butoanelor acestuia, ceea ce a dus la atingerea obiectivelor inițiale.

Pe parcursul lucrării m-am confruntat cu probleme tipice învățării automate, precum \emph{overfitting-ul}, și am învățat importanța datelor și a procesării acestora.
Aceste lucruri se extind și la partea de antrenare și la versionarea modelelor antrenate, pentru a avea evidența evoluției performanței.
Documentarea acestor modele și a fiecărei schimbare este foarte importantă, lucru pe care inițial nu l-am valorificat suficient de mult.

Am realizat și un experiment în care am încercat să reconstruiesc funcționalitatea unei biblioteci Python pe care m-am bazat, care ajută la identificarea reperelor faciale.
Deși experimentul nu a avut rezultate promițătoare, am dobândit mai multe cunoștințe despre metodele \emph{state of the art} de a localiza reperele faciale și despre cum arată procedeul de a realiza asta.
Acest lucru m-a învățat și necesitatea de ``antrenament'' în a citi mai multe lucrări științifice.

În concluzie, această lucrare de licență mi-a permis înglobarea aptitudinilor și cunoștințelor dobândite ca student dar, în același timp, a pus în lumină și lucrurile pe care trebuie să le îmbunătățesc.
Aplicația poate fi îmbunătățită în continuare prin adresarea lipsurilor, precum funcționalități de apăsare dublă/lungă a butoanelor mouse-ului.