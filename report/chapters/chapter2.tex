\chapter{Informații preliminare}
\section{Maniera de lucru}
\paragraph{}
Pentru dezvoltarea, structurarea și versionarea codului lucrării, am folosit platforma gratuită Github\footnote{\url{https://github.com}}.
Repository-ul proiectului poate fi accesat la această adresă: \url{https://github.com/sergiuiacob1/iClicker/}.

\section{Configurație utilizată}
\paragraph{}
Pentru rezultatele prezentate aici, este important de cunoscut că au fost create folosind următorul laptop: ``MacBook Pro A1502, Early 2015''.
Acest laptop dispune de un webcam capabil de o rezoluție maximă HD (1280x720).
Experimentele realizate au fost făcute în general în medii bine luminate, întrucât o dată cu scăderea intensității luminii suferă și utilitatea aplicației prezentate aici.

\section{Informații tehnice}
\paragraph{}
Unul dintre cele mai populare limbaje de programare când vine vorba de Învățare Profundă este \emph{Python}.
Astfel, am ales să dezvolt aplicația folosind acest limbaj, deoarece suportul din partea comunității este unul foarte bun și resursele găsite online pentru a rezolva probleme comune sunt vaste.
Aici este o listă a tehnologiilor pe care le-am folosit impreună cu Python:

\begin{center}
    \begin{tabular}{ c c c }
        Nume & Versiune & Link-uri utile \\
        \hline
        Python3 & 3.7 & \url{https://www.python.org} \\
        Conda & 4.8.0 & \url{https://conda.io/} \\
    \end{tabular}
\end{center}

\paragraph{}
TODO asta trebuie sa dispara de aici. De pus in README pe github.
Pt fiecare biblioteca sa spun la ce am folosit-o.

\paragraph{}
Trebuie să precizez, de asemenea, câteva biblioteci din Python care au adus funcționalități cruciale acestui proiect.
Una dintre ele este \emph{OpenCV}, pe care am folosit-o pentru captura de imagini.
Pentru a construi o interfață grafică am folosit \emph{PyQt5}.

\begin{center}
    \begin{tabular}{ c c c }
        Nume bibliotecă & Versiune & Link-uri utile \\
        \hline
        Keras & 2.2.4 & \url{https://keras.io} \\
        PyTorch & 1.4 & \url{https://pytorch.org} \\
        OpenCV & 4.1.2 & \url{https://opencv.org} \\
        PyQt5 & 5.14 & \url{https://pypi.org/project/PyQt5/} \\
        dlib & 19.19.0 & \url{https://pypi.org/project/dlib/} \\
        imutils & 0.5.3 & \url{https://pypi.org/project/imutils/} \\
    \end{tabular}
\end{center}

\section{Limite și constrângeri}
\paragraph{}
Aplicația are niște limite și lucrează de asemenea cu niște presupuneri, precum faptul că utilizatorul folosește un singur monitor și un singur webcam.
De asemenea, imaginile folosite sunt cu mine însumi, deci trebuie luat acest lucru pentru orice rezultat prezentat.

\paragraph{}
Aplicația este menită să se poată ``mula'' pe fizionomia utilizatorului, însă este posibil să aibă performanțe mai slabe pentru persoanele care poartă ochelari.
Motivul pentru care se întâmplă acest lucru este acela că aplicația lucrează cu ochii utilizatorului, iar dacă lumina se reflectă în lentilele ochelarilor, ochii ar putea fi indistinctibili.
Ca o ultimă mențiune, aplicația se concentrează majoritar pe poziția pupilelor relativ la ochi (glob ocular + anexe ale globului ocular), deci se va considera că poziția capului nu va suferi schimbări majore între datele de antrenament și datele de test.