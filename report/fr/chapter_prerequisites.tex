\section{Manière de travailler}
\paragraph{}
Pour développer le projet, j'utiliserai tout d'abord GitHub pour structurer et sauvegarder le code.
Vous pouvez trouver le dépôt du projet ici : \href{https://github.com/sergiuiacob1/iClicker/}{https://github.com/sergiuiacob1/iClicker/}
% To develop the project, I will use, first of all, GitHub to structure and save the code.
% You may find the project repository here: \href{https://github.com/sergiuiacob1/iClicker/}{https://github.com/sergiuiacob1/iClicker/}

\section{Configuration utilisée}
\paragraph{}
Pour chacun des résultats présentés ici, il est important de savoir que les expériences ont été réalisées sur la machine suivante: ``MacBook Pro A1502, début 2015''.
CPU: Core i5 (I5-5287U), 2.9GHz. RAM: 16 GB, LPDDR3 à 1866 MHz.
Ce portable n'a pas de GPU, donc les modèles d'entraînement prendront plus de temps.
% For any of the results presented here, it is important to know that the experiments were done on the following machine: ``MacBook Pro A1502, Early 2015''.
% CPU: Core i5 (I5-5287U), 2.9GHz. RAM: 16GB 1866 MHz LPDDR3.
% This laptop does not have a GPU, so training models will take a longer time.
\paragraph{}
La caméra de l'ordinateur portable n'est pas géniale, mais elle est capable de prendre des images HD, avec une résolution allant jusqu'à 1280x720.
Cependant, elle ne fonctionne pas bien dans de mauvaises conditions d'éclairage, je vais donc essayer d'obtenir les données dans de bonnes conditions d'éclairage.
% The laptop's camera isn't a great one, but it is capable of taking HD images, with a resolution of up to 1280x720.
% However, it does not do well in poor lighting, so I will try to get the data in good lighting conditions.

\section{Informations techniques}
\paragraph{}
L'un des langages de programmation les plus populaires pour les problèmes d'Apprentissage Profond est Python.
J'ai donc choisi de développer ce projet en Python, car le soutien de la communauté et les ressources trouvées en ligne sont vastes.
Voici une liste des technologies que j'ai utilisées:
% One of the most popular programming languages of choice when it comes to Deep Learning problems is Python.
% I have, therefore, chosen to develop this project in Python, since the community support and the resources found online are vast.
% Here's a list of the technologies I have used:

\begin{center}
    \begin{tabular}{ c c c }
        Technology name & Version & Useful links \\
        \hline
        Python3 & 3.7 & \url{https://www.python.org} \\
        Conda & 4.8.0 & \url{https://conda.io/} \\
    \end{tabular}
\end{center}

\paragraph{}
Je dois également noter quelques bibliothèques Python importantes qui ont des fonctionnalités cruciales pour ce projet.
L'une d'entre elles est OpenCV, que j'utiliserai pour la détection des visages et la capture d'images.
PyQt5 a été utilisé pour construire une interface graphique pour l'application, afin qu'elle soit plus conviviale.
% I also have to note some important Python libraries that have some crucial functionalities for this project.
% One of them is OpenCV, which I will use for face detection and for capturing images.
% PyQt5 was used to build a graphical interface for the app, so it can be more user friendly.

\begin{center}
    \begin{tabular}{ c c c }
        Library name & Version & Useful links \\
        \hline
        Keras & 2.2.4 & \url{https://keras.io} \\
        PyTorch & 1.4 & \url{https://pytorch.org} \\
        OpenCV & 4.1.2 & \url{https://opencv.org} \\
        PyQt5 & 5.14 & \url{https://pypi.org/project/PyQt5/} \\
        dlib & 19.19.0 & \url{https://pypi.org/project/dlib/} \\
        imutils & 0.5.3 & \url{https://pypi.org/project/imutils/} \\
    \end{tabular}
\end{center}

\section{Limites}
\paragraph{}
Au moment où nous écrivons ces lignes, les limites actuelles sont les suivantes: l'application ne fonctionne qu'avec un moniteur et une caméra.
En outre, je n'ai pas encore étudié comment cela devrait fonctionner si l'utilisateur porte des lunettes, je suppose donc qu'il n'en porte pas.
Les résultats présentés sont basés sur les données que j'ai recueillies, c'est-à-dire des images de moi-même regardant le moniteur.
De plus, il n'a été testé que sur MacOS, mais il devrait fonctionner sans problème sous Linux également.
% At the moment of writing this, the current limits are: the application only works with one monitor and one camera.
% Also, I have not yet looked into how this should work if the user is wearing glasses, so I will assume that he is not.
% The results presented are based on the data that I collected, that is images of myself looking at the monitor.
% Also, it was only tested on MacOS, but should work on Linux as well without problems.
